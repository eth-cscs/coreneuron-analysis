In this section the implementation of the code that forms and solves the matrix, which accounts for 99\% of the time to solution will be described. The algorithm themselves are quite simple, however their implementation is very difficult to understand. To make it easier to understand to understand implementation, the core routines have been rewritten in a high-level pseudo code similar to Julia.

An example of the pseudo code represents the following C code
\begin{shaded}
\begin{lstlisting} [language=C,breaklines=true]
for (tml = _nt->tml; tml; tml = tml->next)
  if (memb_func[tml->index].current) {
    mod_f_t s = memb_func[tml->index].current;
    (*s)(_nt, tml->ml, tml->index);
  }
\end{lstlisting}
\end{shaded}
\noindent as
\begin{shaded}
\begin{lstlisting} [language=julia,breaklines=true]
for mechanism in thread.mechanisms
  current(mechanism)
end
\end{lstlisting}
\end{shaded}
\noindent Note that the same level of clarity is possible with C+11:
\begin{shaded}
\begin{lstlisting} [language=julia,breaklines=true]
for( auto &mechanism : thread.mechanisms() ) {
  mechanism.current(thread);
}
\end{lstlisting}
\end{shaded}

The inner part of each time step is implemented in the function \lst{nrn_fixed_step_thread()}, in \file{nrnoc/fadvance_core.c}. The routine takes as its argument a pointer to a struct of type \lst{NrnThread}, see \fig{lst:NrnThread}, which holds state relating to a set of cells to be integrated in time.
\begin{shaded}
\lstinputlisting [language=julia,breaklines=true] {./code/fixed_step_thread.jl}
\end{shaded}

A breakdown of wall time for the steps in \lst{nrn_fixed_step()} is given in \fig{fig:calltree}. Some of the routines listed here have less than 1\% of wall time (including the linear system solve in \lst{nrn_solve_minimal()}), however they are discussed below because they access they have implementation details that will influence the implementation on many-core architectures (e.g. GPU and MIC).

%%%%%%%%%%%%%%%%%%%%%%%%%%%%%%%%%%%%%%%%%%%%%%%%%%%%%%%%%%%%%%
\subsubsection{Building matrix and RHS: \lst{setup_tree_matrix()}}
%%%%%%%%%%%%%%%%%%%%%%%%%%%%%%%%%%%%%%%%%%%%%%%%%%%%%%%%%%%%%%
The function \lst{setup_tree_matrix()} generates the diagonal, the $d_i$ values, and the RHS vector. These tasks are performed in two separate routines, \lst{nrn_lhs()} and \lst{nrn_rhs()}.
\begin{shaded}
\lstinputlisting [language=julia,breaklines=true] {./code/setup_tree_matrix.jl}
\end{shaded}

Points
\begin{itemize}
\item
    The array \lst{p} is an index array containing the parent node indexes.
\item
    The arrays \lst{VEC_*} correspond to the vectors $\vv{a}, \vv{b}, \vv{d}, \vv{v}, \vv{r}$ that define the linear system.
\item
    Each thread has multiple cells, each with their own tree representation. The cells are packed together, with the root node of each cell placed first in the list of all nodes, hence the definition of \lst{child_nodes} excluding indexes $1:ncells$.
\item
    Nearly all (i.e. 99\%) of the time in these two routines is spent in the calls to the \lst{mechanism.current()} and \lst{mechanism.jacob()} routines.
\item
    The matrix updates still must be considered, because there are potential race conditions in a multi-threaded/GPU implementation. For example the statement \\ \lst{VEC_RHS[p[i]] += dv * VEC_A[i]} \\ will lead to a race condition if two threads with the same parent node try to update the RHS vector at the same time.
\end{itemize}

The \lst{mechanism.current()} and \lst{mechanism.jacob()} routines are defined in the \file{/mech/cfiles} path, and are automatically generated from Neuron \hoc DSL. \fig{fig:calltree} shows that all of the computational work in the \neuron benchmark used in this report is performed by functions from the \hoc layer. The \lst{jacob} functions are also very simple, and all have the same form \hilight{(I think, there may be some exceptions)}
\begin{shaded}
\lstinputlisting [language=julia,breaklines=true] {./code/jacob.jl}
\end{shaded}

The \lst{mechanism.current} calls contribute 45\% of time to solution for the TEST2 benchmark. They are uniquely defined for each mechanism. \hilight{Is it possible to present a few different examples that have all the expected \emph{patterns} in a current implementation?}. A ``representative'' example of a \lst{current} implementation is:
\begin{shaded}
\lstinputlisting [language=julia,breaklines=true] {./code/current.jl}
\end{shaded}
\hilight{I don't understand the \lst{ppvar} variable, and how it is used to implement lookups like \lst{_ion_dinadv}. Any insight here would be appreciated.}

The \lst{nrn_cap_jacob()} function is a very simple, illustrating a common data access pattern whereby a vector (int this case \lst{VEC_D}) is updated according to the parent node if the loop index:
\begin{shaded}
\lstinputlisting [language=julia,breaklines=true] {./code/nrn_cap_jacob.jl}
\end{shaded}

%%%%%%%%%%%%%%%%%%%%%%%%%%%%%%%%%%%%%%%%%%%%%%%%%%%%%%%%%%%%%%
\subsubsection{Solving the linear system: \lst{nrn_solve_minimal()}}
%%%%%%%%%%%%%%%%%%%%%%%%%%%%%%%%%%%%%%%%%%%%%%%%%%%%%%%%%%%%%%
The solution of the linear system using Hines algorithm is straightforward, and is implemented in \file{/nrnoc/solve_core.c}.

\begin{shaded}
\lstinputlisting [language=julia,breaklines=true] {./code/nrn_solve_minimal.jl}
\end{shaded}

%%%%%%%%%%%%%%%%%%%%%%%%%%%%%%%%%%%%%%%%%%%%%%%%%%%%%%%%%%%%%%
\subsubsection{Advancing the solution: \lst{update()}}
%%%%%%%%%%%%%%%%%%%%%%%%%%%%%%%%%%%%%%%%%%%%%%%%%%%%%%%%%%%%%%

These routines have almost no computational overhead, and are included here for completeness.

The solution to the linear system, stored in \lst{VEC_RHS} is actually the delta in solution, that is $\at{\text{RHS}} = \at{V}^{n+1} - \at{V}^{n}$. The \lst{update()} function updates the solution in \lst{VEC_V} by adding the contribution in \lst{VEC_RHS}. The \lst{update()} function is the only part of the \neuron code that successfully vectorizes, because of the stride-one data access pattern in the update.

\hilight{I don't know the purpose of the \lst{second_order_cur()} and \lst{nrn_capacity_current()} routines.}

\begin{shaded}
\lstinputlisting [language=julia,breaklines=true] {./code/update.jl}
\end{shaded}
%%%%%%%%%%%%%%%%%%%%%%%%%%%%%%%%%%%%%%%%%%%%%%%%%%%%%%%%%%%%%%
\subsubsection{Updating \emph{state}: \lst{nonvint()}}
%%%%%%%%%%%%%%%%%%%%%%%%%%%%%%%%%%%%%%%%%%%%%%%%%%%%%%%%%%%%%%
The \lst{nonvint()} function is a simple lookup of the \lst{state()} function defined for each mechanism. These calls are a significant contribution to computational overheads -- greater than 40\% for the TEST2 benchmark.

\begin{shaded}
\lstinputlisting [language=julia,breaklines=true] {./code/nonvint.jl}
\end{shaded}

The \lst{state()} function implementations are derived from the \hoc implementation. These have obvious potential for vectorization, because they do not appear to have the ``parent update'' pattern in other loops. However this would require using structore of array (SoA) storage (see the next section). Below is an example of one state update -- note the many exponentials (some of which are redundant, i.e. )

\begin{shaded}
\lstinputlisting [language=julia,breaklines=true] {./code/state.jl}
\end{shaded}
