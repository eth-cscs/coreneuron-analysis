The code in its current form is a mixture of C and C++.

It must be noted that the code base is currently very challenging to understand and benchmark. It is derived directly from the Bluron/Neuron code base, which has grown organically over a period of more tha 20 years. As such there are very many opportunities to simplify and improve the code using modern programming languages and development techniqies.

To use both CPUs and accelerators (e.g. GPU and MIC) effectively, parts of the code will have to be refactored, or rewritten significantly. Rewriting will be challenging, given the difficulty of understanding the current code, which is a prerequisite for rewriting the code. Developing documentation of the algorithms as they are currently implemented is essential if the code is to be refactored.

From early work with the code base, the majority of wall time for the example circuits release for the PCP is spent in a relatively small set of code: the \lst{nrnoc} solver implementation, and the mechanisms defined in \lst{/mecb/cfiles}. The code in its current form might be difficult to reason about, however the algorithms themselves are relatively straightfoward.

An aim of this report is as a first attempt at describing the algorithms clearly, to make it possible to reason about how them without getting distracted by implementation details. To assist in this, many of the algorithms are presented in a pseudo-language similar to \emph{Julia}\footnote{See the website: \file{julialang.org}}, which should be familiar to users familiar with \emph{Matlab}.

%-------------------------------------------------------------
\subsection{Code Issues}
%-------------------------------------------------------------
Some of the anti-patterns that make understanding the code difficult are:
\begin{itemize}
    \item
        There are many global variables. With many static symbols used for scoping within a translation units, which makes understanding where variables are defined and used challenging.
    \item
        The use of preprocessor \lst{#define} makes the code very difficult to reason about. Many variable names are redefined, which obscures the meaning of the code. There are also many ``magic numbers'' that are defined on differently in different translation units.
    \item
        The C++ code uses new and delete keywords, sometimes forgetting to delete. Should use static arrays or STL containers depending on needs.
    \item
        overly complicated data structures that make it very difficult for both humans and compilers to reason about the code. As a result there are a lot of lost opportunities for optimization for the compiler.
    \item
        Poor (non-existant in places) comments and naming of variables.
\end{itemize}

%-------------------------------------------------------------
\subsection{Code Layout}
%-------------------------------------------------------------
The source code is packaged in a file \lst{CoreBluron.tar.gz}, which has the directory structure in \fig{fig:DirectoryStructure}.
In terms of time to solution, functions defined in \lst{mech/cfiles} dominate. These are called from the solver routines in \lst{nrnoc}, which implements the core computation, and is the focus of the analysis here.

\begin{figure}[tp!]
%---------------------------
\fbox{ \parbox{\textwidth} {

\begin{itemize}
    %%%%%%%%%%%%%%%%%%%%%%%%%%%%%%%%%%%%%%%%%
    \item \textbf{nrniv}
    \begin{itemize}
        \item C and C++ (11,470 lines).
        \item The \neuron driver: \lst{main()} is in \lst{main1.cpp}.
    \end{itemize}

    %%%%%%%%%%%%%%%%%%%%%%%%%%%%%%%%%%%%%%%%%
    \item \textbf{nrnoc}
    \begin{itemize}
        \item C (2,889 lines).
        \item The \neuron ``engine''
        \begin{itemize}
            \item storage
            \item solvers
            \item time stepping
        \end{itemize}
    \end{itemize}

    %%%%%%%%%%%%%%%%%%%%%%%%%%%%%%%%%%%%%%%%%
    \item \textbf{mech/cfiles}
    \begin{itemize}
        \item C (11,301 lines).
        \item Definitions of all the mechanisms.
        \item generated from \hoc files by Neuron.
        \item the generated code is very messy (use \lst{clang-format} to make things bearable)
    \end{itemize}

    %%%%%%%%%%%%%%%%%%%%%%%%%%%%%%%%%%%%%%%%%
    \item \textbf{nrnmpi}
    \begin{itemize}
        \item C (1,096 lines).
        \item Wrappers around MPI routines.
        \item Spike exchange implementation.
        \item Global variables that store MPI state.
    \end{itemize}

    %%%%%%%%%%%%%%%%%%%%%%%%%%%%%%%%%%%%%%%%%
    \item \textbf{utils}
    \begin{itemize}
        \item C++ (4,494 lines)
        \item Random number generators
    \end{itemize}
\end{itemize}

}}
%---------------------------

\caption{Overview of the directory structure for the source code. There are a total of 31,250 lines of code (including white space and comments).}
\label{fig:DirectoryStructure}

\end{figure}

%%%%%%%%%%%%%%%%%%%%%%%%%%%%%%%%%%%%%%%%%%%%%%%%%%%%%%%%%%%%%%
\subsection{Building}
%%%%%%%%%%%%%%%%%%%%%%%%%%%%%%%%%%%%%%%%%%%%%%%%%%%%%%%%%%%%%%
The code was built on Cray XC-30 system Piz Daint at CSCS with minimal fuss using the GNU toolchain.
The Cray compiler toolchain had problems that are not insurmountable, but they would require a lot of tinkering with the \emph{Buildyard}\footnote{\file{github.com/Eyescale/Buildyard}} build tool used by \neuron.

The Buildyard uses cmake, with a custom set of cmake modules developed by BBP (the BuildYard modules). Many of these modules are not required by \neuron, and configuration can be sped up significantly by removing them (for example the C++11 tests). The cmake configuration attempts to determine the version of the Cray compiler by passing a flag that the compiler doesn't recognise, causing the configuration to exit.

%%%%%%%%%%%%%%%%%%%%%%%%%%%%%%%%%%%%%%%%%%%%%%%%%%%%%%%%%%%%%%
\subsection{Datasets}
%%%%%%%%%%%%%%%%%%%%%%%%%%%%%%%%%%%%%%%%%%%%%%%%%%%%%%%%%%%%%%
There are two data sets provided with the PCP benchmark code:
\begin{itemize}
    \item \textbf{TEST1\_CACHE} A network small enough to fit into Cache of one rack of BG/Q. Has size of 2.5G on disk.
    \item \textbf{TEST2\_DRAM} A much larger network (size 4.5T on disk), that fits in DRAM of one rack of BG/Q.
\end{itemize}

%%%%%%%%%%%%%%%%%%%%%%%%%%%%%%%%%%%%%%%%%%%%%%%%%%%%%%%%%%%%%%

