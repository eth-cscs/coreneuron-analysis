The code in its current form is a mixture of C and C++.

It must be noted that the code base is currently very challenging to understand and benchmark. It is derived directly from the Bluron/Neuron code base, which has grown organically over more than 20 years. There are very many opportunities to simplify and improve the code using modern programming languages and development techniques.

To use both CPUs and accelerators (e.g. GPU and MIC) effectively, parts of the code will have to be refactored or rewritten. The main challenge in refactoring the code is not the complexity of the algorithms, it is the difficulty of understanding the current code. Developing documentation of the algorithms as they are currently implemented is an essential first step.

From early work with the code base, the majority of wall time for the example circuits release for the PCP is spent in a relatively small set of code: the \lst{nrnoc} solver implementation, and the mechanisms defined in \lst{/mech/cfiles}. Improving performance, portability and maintainability will benefit from working on the core computational API.

An aim of this report is as a first attempt at describing the algorithms clearly, to make it possible to reason about how them without getting distracted by implementation details. To assist in this, many of the algorithms are presented in a pseudo-language similar to \emph{Julia}\footnote{See the website: \file{julialang.org}}, which should be familiar to users familiar with \emph{Matlab}.

%-------------------------------------------------------------
\subsection{Understanding The Code}
%-------------------------------------------------------------
Neuron has grown organically, with new features ``bolted on'' over a long period of time (e.g. support for Python scripting, threading, etc.).
Some of these features have been removed in \neuron, leaving behind some design patterns that at first confuse newcomers.
One consequence is that there are some data structures that are overly-complicated for the role they currently serve, but they were part of a more complicated design before the \neuron rewrite.
There are also many instances where the naming of variables, global variables, and gratuitous use of the preprocessor don't help.
\begin{note}
It is recommended that the code be reformatted using a tool like \file{clang-format} (especially the automatically generated C files in \file{mech/cfiles}). Applying the preprocessor to individual files shows the origin of symbols that have been \#defined with the preprocessor. For example, using the ``\file{-E -P}'' flag in \texttt{gcc} (\lst{-E} applies the preprocessor, and \lst{-P} makes the output more human readable.)
\end{note}

%-------------------------------------------------------------
\subsection{Code Layout}
%-------------------------------------------------------------
The source code is packaged in a file \lst{CoreBluron.tar.gz}, which has the directory structure in shown below.
In terms of time to solution, functions defined in \lst{mech/cfiles} dominate. These are called from the solver routines in \lst{nrnoc}, which implements the core computation, and is the focus of the analysis here.

\begin{infobox}{The directory structure of the \neuron distributed as part of the PCP}
\begin{itemize}[leftmargin=*]
    %%%%%%%%%%%%%%%%%%%%%%%%%%%%%%%%%%%%%%%%%
    \item \textbf{nrniv}
    \begin{itemize}
        \item C and C++ (11,470 lines).
        \item The \neuron driver: \lst{main} function is in \lst{main1.cpp}.
    \end{itemize}

    %%%%%%%%%%%%%%%%%%%%%%%%%%%%%%%%%%%%%%%%%
    \item \textbf{nrnoc}
    \begin{itemize}
        \item C (2,889 lines).
        \item The \neuron ``engine''
        \begin{itemize}
            \item storage
            \item solvers
            \item time stepping
        \end{itemize}
    \end{itemize}

    %%%%%%%%%%%%%%%%%%%%%%%%%%%%%%%%%%%%%%%%%
    \item \textbf{mech/cfiles}
    \begin{itemize}
        \item C (11,301 lines).
        \item Definitions of all the mechanisms.
        \item generated from \hoc files by Neuron.
        \item the generated code is very messy (use \lst{clang-format} to make things bearable)
    \end{itemize}

    %%%%%%%%%%%%%%%%%%%%%%%%%%%%%%%%%%%%%%%%%
    \item \textbf{nrnmpi}
    \begin{itemize}
        \item C (1,096 lines).
        \item Wrappers around MPI routines.
        \item Spike exchange implementation.
        \item Global variables that store MPI state.
    \end{itemize}

    %%%%%%%%%%%%%%%%%%%%%%%%%%%%%%%%%%%%%%%%%
    \item \textbf{utils}
    \begin{itemize}
        \item C++ (4,494 lines)
        \item Random number generators
    \end{itemize}
\end{itemize}
\end{infobox}

%%%%%%%%%%%%%%%%%%%%%%%%%%%%%%%%%%%%%%%%%%%%%%%%%%%%%%%%%%%%%%
\subsection{Building}
%%%%%%%%%%%%%%%%%%%%%%%%%%%%%%%%%%%%%%%%%%%%%%%%%%%%%%%%%%%%%%
The code was built on Cray XC-30 system Piz Daint at CSCS with minimal fuss using the GNU toolchain.
The Cray compiler toolchain had problems that are not insurmountable, but they would require a lot of tinkering with the \emph{Buildyard}\footnote{\file{github.com/Eyescale/Buildyard}} build tool used by \neuron.

The Buildyard uses cmake, with a custom set of cmake modules developed by BBP (the BuildYard modules). Many of these modules are not required by \neuron, and configuration can be sped up significantly by removing them (for example the C++11 tests). The cmake configuration attempts to determine the version of the Cray compiler by passing a flag that the compiler doesn't recognise, causing the configuration to exit.

%%%%%%%%%%%%%%%%%%%%%%%%%%%%%%%%%%%%%%%%%%%%%%%%%%%%%%%%%%%%%%
\subsection{Datasets}
%%%%%%%%%%%%%%%%%%%%%%%%%%%%%%%%%%%%%%%%%%%%%%%%%%%%%%%%%%%%%%
There are two data sets provided with the PCP benchmark code:
\begin{infobox}{The two data sets available with the PCP}
\begin{itemize}
    \item \textbf{TEST1\_CACHE} A network small enough to fit into Cache of one rack of BG/Q. Has size of 2.5G on disk.
    \item \textbf{TEST2\_DRAM} A much larger network (size 4.5T on disk), that fits in DRAM of one rack of BG/Q.
\end{itemize}
\end{infobox}

%%%%%%%%%%%%%%%%%%%%%%%%%%%%%%%%%%%%%%%%%%%%%%%%%%%%%%%%%%%%%%

