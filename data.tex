%%%%%%%%%%%%%%%%%%%%%%%%%%%%%%%%%%%%%%%%%%%%%%%%%%%%%%%%%%%%%%%%%%%%%%%%%%%%%%
\subsection{Parallel Data Distribution}
%%%%%%%%%%%%%%%%%%%%%%%%%%%%%%%%%%%%%%%%%%%%%%%%%%%%%%%%%%%%%%%%%%%%%%%%%%%%%%
\begin{itemize}
\item
    Each neuron cell is represented as of a tree of nodes, as illustrated in \fig{fig:tree}.
\item
    The properties of individual cells vary significantly, so that the computational resources required to process cells varies.
\item
    The biggest cause of this difference is the distribution of nodes with different mechanism types.
\item
    The \lst{states}/\lst{rates}/\lst{current} kernels of some mechanisms have much higher arithmetic intensity than others.
\item
    The combination of mechanisms in a cell, which varies between different cell types, can be used a-priori estimate it's computational complexity.
\end{itemize}

\begin{itemize}
\item
    To ensure load balancing, the cells are grouped into groups that have roughly equivalent computational overheads during circuit generation.
\item
    Some groups have more cells than others to ensure that total computational effort required per group is balanced.
\item
    Each group of cells is stored in a separate \file{.dat} file, which are then distributed in a round-robin fashion when a \neuron simulation is started.
\end{itemize}

\begin{itemize}
\item
    Parallelism is implemented distributing the cells, with individual cells processed serially.
\item
    There are two levels of parallelism:
    \begin{enumerate}
    \item
        \textbf{MPI}: the cells are distributed between MPI ranks.
    \item
        \textbf{thread}: the cells on each MPI rank are then assigned to a \emph{thread}.
    \end{enumerate}
\item
    Each thread stores the cells assigned to it in a \lst{NrnThread} data structure (see \fig{lst:NrnThread})
\item
    There is one \lst{NrnThread} data structure for each thread.
\item
    Threading is performed using pthreads, with explicit communication of spike information between threads and between MPI processes.
\end{itemize}

%%%%%%%%%%%%%%%%%%%%%%%%%%%%%%%%%%%%%%%%%%%%%%%%%%%%%%%%%%%%%%%%%%%%%%%%%%%%%%
\subsection{Thread Storage}
%%%%%%%%%%%%%%%%%%%%%%%%%%%%%%%%%%%%%%%%%%%%%%%%%%%%%%%%%%%%%%%%%%%%%%%%%%%%%%
Overview of storage of cells in one thread:
\begin{itemize}
\item
    Each thread has multiple cells assigned to it.
\item
    For the TEST2 dataset:
    \begin{itemize}
    \item
        there are around 60--70 cells per thread.
    \item
        each cell has of the order 400--450 nodes.
    \item
        each thread has 25,000--30,000 nodes.
    \end{itemize}
\item
    The nodes for all cells are stored in one flat array
    \begin{itemize}
    \item
        Given have $n_c$ cells and a total of $n$ nodes, the root nodes are indexed \lst{[1:n_c]}, and the rest of the nodes are indexed \lst{[n_c+1:n]}.
    \item
        This is evident in the loops over \lst{child_nodes=ncells+1:nnondes} and \lst{1:ncells}
    \end{itemize}
\end{itemize}

\noindent
Mechanisms in Neuron are implemented using the \hoc DSL, which is translated into C code.
\neuron has the C files allready translated from the \hoc files that are used by BBP\footnote{This reduces the complexity of \neuron, decoupling \neuron from Neuron, which will make it easier modify how the mechanisms are defined.}.
The translated mechanism definitions are in \file{mech/cfiles}, with one mechanism per C source file.
Each file defines functions (like \lst{jacob}, \lst{current}, \lst{alloc}) and meta-data (such as the number of variables required to store a mechanism's state for a node in the tree).
\begin{itemize}
\item
    The mechanism data is stored in global two arrays: \lst{Memb_list memb_list[]} and \lst{Memb_func memb_func[]}.
    \begin{enumerate}
    \item
        The \lst{Memb_func} type has function pointers to the \lst{jacob}, \lst{current}, \lst{state} and other mechanism-specific functions, and other meta-data specific to the mechanism.
    \item
        The \lst{Memb_list} has a pointer to the per-node data, and a list of all the nodes that the mechanism is defined for.
    \end{enumerate}
\item
    The implementation of a mechanism in \lst{mech/cfiles} provides has a function \lst{??} that calls the \lst{register_mech()} function, which adds the mechanisms function callbacks for \lst{jacob} etc, along with meta-data into the global arrays (see \fig{lst:register_mech}).
\item
    Each thread has a list of mechanisms assigned to it, which are accessed via a linked list \lst{NrnThread::mechanisms} (see \fig{lst:NrnThreadInfo} where I have changed the name \lst{tml} to \lst{mechanisms}, to better match the pseudo code.) The linked list indexes the global arrays \lst{memb_list} and \lst{memb_func} \hilight{(why not use an array instead of a linked list?)}.
\item
    There are many opportunities to improve the interface between the runtime (solvers etc) and user-defined mechanisms.
    This model is well-suited to standard object-oriented design.
    Furthermore, much of the meta-data that is currently passed as runtime parameters could be stored as type-information that could help the compiler optimize more agressively.
\end{itemize}

\noindent
Mechanisms and their storage:
\begin{itemize}
\item
    All mechanisms are not applied to different nodes. For example, the \lst{ProbAMPANDMDA_EMS} mechanism will only be applied at a subset of the nodes in a cell.
\item
    Each mechanism has ``state'' that is stored for each node to which it is applied. This state is a set of double-precision values (e.g. a set of values describing the time evolution of a ordinary differential equation).
\item
    The number of state variables varies between mechanisms, ranging from 3 to 35 values.
\item
    Each entry in \lst{memb_list[]} stores 
    \begin{itemize}
    \item
        \lst{int nodecount}: the number of nodes to which
    \item
        \lst{int nodeindices[nodecount]}: the indexes of the nodes to which the mechanism is to be applied.
    \item
        \lst{double data[nodecount*var_per_node]}: AoS storage for the mechanism  values.
    \end{itemize}
\end{itemize}


\begin{figure}
\begin{shaded}
\lstinputlisting [language=C++,breaklines=true] {./code/storage.cpp}
\end{shaded}
\label{lst:NrnThreadInfo}
\caption{The thread (\lst{NrnThread}), mechanism data (\lst{Memb_list}) and mechanism functionality (\lst{Memb_func}) types. I have removed and renamed much of the members to make them better match the pseudo code. Some C++ coding style has also been used.}
\end{figure}

\begin{figure}
\begin{shaded}
\lstinputlisting [language=C++,breaklines=true] {./code/register_mech.cpp}
\end{shaded}
\label{lst:register_mech}
\caption{The routines used in registering a mechanism}
\end{figure}

\begin{figure}[htp!]
\tikzset{
    %Define standard arrow tip
    >=stealth',
    % Define arrow style
    pil/.style={
           ->,
           thick,
           shorten <=2pt,
           shorten >=2pt,}
}
\newcommand{\nodewidth}{1.1cm}
\newcommand{\nodeheight}{0.5cm}
\begin{tikzpicture}[x=0cm, y=0cm, node distance=0 cm,outer sep = 0pt]
\tikzstyle{t0}=[draw, rectangle,  minimum height=\nodeheight, minimum width=\nodewidth, fill=red!10,anchor=south west]
\tikzstyle{t1}=[draw, rectangle,  minimum height=\nodeheight, minimum width=\nodewidth, fill=green!10,anchor=south west]
\tikzstyle{t2}=[draw, rectangle,  minimum height=\nodeheight, minimum width=\nodewidth, fill=blue!10,anchor=south west]
\tikzstyle{t3}=[draw, rectangle,  minimum height=\nodeheight, minimum width=\nodewidth, fill=yellow!20,anchor=south west]
\tikzstyle{index}=[draw, rectangle,  minimum height=\nodeheight, minimum width=\nodewidth, fill=black!10,anchor=south west]
\tikzstyle{blank}=[draw=none, fill=none, rectangle,  minimum height=\nodeheight, minimum width=1cm, anchor=south west]
\tikzstyle{over}=[draw=black, rectangle,  minimum height=\nodeheight, minimum width=3.3cm, anchor=south west]
\tikzstyle{pointer}=[draw=black!50, rectangle,  minimum height=\nodeheight, minimum width=2.5cm, anchor=south west]

\node[pointer] (dataAoS) at (0,0)         {\lst{dataAoS}};
\node[t0] (m00) [right = 1cm of dataAoS] {$m_{0,0}$}
    edge[pil,<-] (dataAoS.east);
\node[t1] (m01) [right = of m00] {$m_{0,1}$};
\node[t2] (m02) [right = of m01] {$m_{0,2}$};
\node[over] (over0) [below = of m01] {node 0 params};
\node[t0] (m10) [right = of m02] {$m_{1,0}$};
\node[t1] (m11) [right = of m10] {$m_{1,1}$};
\node[t2] (m12) [right = of m11] {$m_{1,2}$};
\node[over] (over1) [below = of m11] {node 4 params};
\node[t0] (m20) [right = of m12] {$m_{2,0}$};
\node[t1] (m21) [right = of m20] {$m_{2,1}$};
\node[t2] (m22) [right = of m21] {$m_{2,2}$};
\node[blank] (blank) [right = of m22] {...};
\node[over] (over2) [below = of m21] {node 11 params};
\node[t0] (m30) [right = of blank] {$m_{n,0}$};
\node[t1] (m31) [right = of m30] {$m_{n,1}$};
\node[t2] (m32) [right = of m31] {$m_{n,2}$};\\
\node[over] (over3) [below = of m31] {node 87 params};

\node[index] (i0) [above = 0.5 cm of m00]   {$0$};
\node[index] (i1) [right = of i0]   {$4$};
\node[index] (i2) [right = of i1]   {$11$};
\node[blank] (blanki) [right = of i2] {...};
\node[index] (in) [right = of blanki]   {$87$};
\node[pointer] (indicesAoS) [left = 1cm of i0]   {\lst{nodeindices}}
    edge[pil,->] (i0.west);

\node[index] (nnodes) [above = 0.5 cm of i0]   {$n$};
\node[pointer] (numnodes) [left = 1cm of nnodes]   {\lst{nodecount}}
    edge[pil,-] (nnodes.west);

\node[t0]  (00) [below = 1 cm of m00] {$m_{0,0}$};
\node[t0]  (10) [right = of 00] {$m_{1,0}$};
\node[t0]  (20) [right = of 10] {$m_{2,0}$};
\node[blank] (blank0) [right = of 20] {...};
\node[t0]  (n0) [right = of blank0] {$m_{n,0}$};

\node[t1]  (01) [below = 1.7 cm of m00] {$m_{0,1}$};
\node[t1]  (11) [right = of 01] {$m_{1,1}$};
\node[t1]  (21) [right = of 11] {$m_{2,1}$};
\node[blank] (blank1) [right = of 21] {...};
\node[t1]  (n1) [right = of blank1] {$m_{n,1}$};

\node[t2]  (02) [below = 2.4 cm of m00] {$m_{0,2}$};
\node[t2]  (12) [right = of 02] {$m_{1,2}$};
\node[t2]  (22) [right = of 12] {$m_{2,2}$};
\node[blank] (blank2) [right = of 22] {...};
\node[t2]  (n2) [right = of blank2] {$m_{n,2}$};

\node[pointer] (indicesSoA) [left = 1cm of 01]   {\lst{dataSoA[]}}
    edge[pil,->,anchor=east] (00.west)
    edge[pil,->] (01.west)
    edge[pil,->] (02.west);

\end{tikzpicture}

\caption{The current AoS layout of mechanism parameters for all applicable nodes.}
\end{figure}

\noindent
Observations
\begin{itemize}
\item
    Splitting the cells into seperate, thread-specific, data structures complicates the code. This feels like it was added at some point to facilicate threading.
\item
    Could be stripped away, to store all cells on a node/numa-region/device into a single pool.
\item
    The AoS storage is inefficient:
    \begin{itemize}
    \item
        It doesn't vectorize (see \fig{fig:papisample}).
    \item
        Poor cache/bandwidth utlization for loops (such as the \lst{jacob} update) where only one or two data values in a mechanism are touched. For each 64 byte cache line loaded, only 8 bytes of 64 are used.
    \end{itemize}
    An SoA storage would address both issues.
\item
    With SoA vectorization of most loops would still not be possible, because of the gather/scatter implicit in using the node indexes to read/write to the V and RHS vectors.
    \begin{itemize}
    \item
        Perform scatter before computing the current, then gather after.
    \end{itemize}
\end{itemize}

