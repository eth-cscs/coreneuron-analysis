%-------------------------------------------------------------------------------
\begin{table}[htp!]
    \centering
%-------------------------------------------------------------------------------
\begin{tabular}{l|l|rrrrr}
\multirow{2}{*}{}
& cells-per-core  &    1  & 2     & 4     & 8  & 16  \\
& cells-per-node  &    8  & 16    & 32    & 64 & 128 \\
\hline
\multirow{8}{*}{nodes}
&1                & 390.8 & 383.6 & 381.6 & 380.5 & 381.3\\
&2                & 390.9 & 385.1 & 384.0 & 385.3 & 382.5\\
&4                & 394.8 & 392.7 & 389.9 & \textcolor{blue}{194.1} & 382.6\\
&8                & 400.4 & 392.8 & 385.9 & 382.7 & 381.0\\
&16               & 401.9 & 394.3 & 388.6 & 384.2 & 381.8\\
&32               & 401.7 & 393.6 & 387.0 & 383.8 & \textcolor{red}{DNF}\\
&64               & 403.0 & 395.7 & 390.0 & 385.4 & 382.3 \\
&128              & 411.0 & 396.1 & 390.4 & 385.5 & 378.7 \\
&256              & 411.4 & 397.0 & 390.7 & \textcolor{blue}{195.5} & 374.9\\
&512              & 413.5 & 396.8 & 389.2 & 377.7 & --\\
%\hline
\end{tabular}

%-------------------------------------------------------------------------------
\label{tbl:test2scaling}
\caption{Wall time for TEST2 as both the number of nodes and cells-per-node are varied. Each test always use 8 cores/MPI ranks per-node. The times in \textcolor{blue}{blue} indicate unexplained timings, and those marked \textcolor{red}{DNF} did not finish due to memory restrictions (it is not possible to run with more than 16 cells per node due to memory restrictions.)}
\end{table}
%-------------------------------------------------------------------------------

\todo{describe adding Papi counters to instrument parts of code. Show basic timing results in \fig{fig:calltree}.}

\begin{figure}[htp!]
\centering
\includegraphics[width=\textwidth]{./images/calltree.pdf}
\caption{Calltree and percentage of wall time contribution for the main computational algorithm. Branches marked in blue indicate a significant contribution to wall time.}
\label{fig:calltree}
\end{figure}


