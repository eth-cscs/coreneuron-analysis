Here we present preliminary benchmarking results based on the benchmark datasets distributed with the PCP benchmark. The benchmark data is untarred to the paths \file{VENDORS/TEST1_CACHE} and \file{VENDORS/TEST2_DRAM}, which contain data files \file{[0-9]*.dat} that define $2^{14}=16,384$ and $2^{15}=32,768$ individual cells respectively. The number of cells to use in a simulation is set in the first line of the file \file{files.dat}. The cells are assigned to individual MPI ranks in a round-robin fashion, e.g. 32 files with 8 MPI ranks will assign $32/8=4$ cells per rank.

\tbl{tbl:test2scaling} shows the wall time for performing 100 time steps with the TEST2 dataset as the number of MPI ranks and the number of cells-per-rank are varied on Piz Daint, which has a eight-core Sandy Bridge socket on each node.
\begin{itemize}
\item
    There are 8 MPI ranks per node, 1 per core, so the number of MPI ranks in 8$\times$ the number of nodes.
\item
    The amount of memory available per node, 32\,GB limits the number of cells to 128 per node (16 per MPI rank).
\item
    Scaling was performed from 1 to 512 nodes (8 to 4096 MPI ranks).
    \begin{itemize}
    \item
        With one cell per core, there is an efficiency of 95.6\% from 1--512 nodes
    \item
        As the number of cells per core is increased both wall time per cell and scaling improve, indeed it is slightly better than perfect at 100.1\% for 1--512 nodes and 8 cells per core.
    \end{itemize}
\item
    There are some anomilies, with some two simulations (marked in blue) with 8 cells per core finishing in half the time of the others, and one simulation with 16 cells per core failing because it ran out of memory.
    \begin{itemize}
    \item
        The cells distributed amongst the cores are not all equal in size, so some variability is expected.
    \item
        Modifying the total number of cells in a simulation will change the number of connections between neurons, which will modify behaviour.
    \end{itemize}
\item
    From this informal analysis, scaling from 1 to 512 nodes appears to be very good.
    \begin{itemize}
    \item
        Communication overheads do not appear to be significant.
    \item
        The focus should be on in-node performance.
    \end{itemize}
\end{itemize}

%-------------------------------------------------------------------------------
\begin{mytable}{simple coreneuron scaling test}{tbl:test2scaling}
\begin{center}
\begin{tabular}{l|l|rrrrr}
\multirow{2}{*}{}
& cells-per-core  &    1  & 2     & 4     & 8  & 16  \\
& cells-per-node  &    8  & 16    & 32    & 64 & 128 \\
\hline
\multirow{8}{*}{nodes}
&1                & 390.8 & 383.6 & 381.6 & 380.5 & 381.3\\
&2                & 390.9 & 385.1 & 384.0 & 385.3 & 382.5\\
&4                & 394.8 & 392.7 & 389.9 & \textcolor{red}{194.1} & 382.6\\
&8                & 400.4 & 392.8 & 385.9 & 382.7 & 381.0\\
&16               & 401.9 & 394.3 & 388.6 & 384.2 & 381.8\\
&32               & 401.7 & 393.6 & 387.0 & 383.8 & \textcolor{red}{DNF}\\
&64               & 403.0 & 395.7 & 390.0 & 385.4 & 382.3 \\
&128              & 411.0 & 396.1 & 390.4 & 385.5 & 378.7 \\
&256              & 411.4 & 397.0 & 390.7 & \textcolor{red}{195.5} & 374.9\\
&512              & 413.5 & 396.8 & 389.2 & 377.7 & --\\
%\hline
\end{tabular}
\end{center}
\vspace{10pt}
Wall time for TEST2 as both the number of nodes and cells-per-node are varied. Each test always use 8 cores/MPI ranks per-node. The times in \textcolor{red}{red} indicate unexplained timings, and those marked \textcolor{red}{DNF} did not finish due to memory restrictions (it is not possible to run with more than 16 cells per node due to memory restrictions.)
\end{mytable}
%*******************************************************************************

%%%%%%%%%%%%%%%%%%%%%%%%%%%%%%%%%%%%%%%%%%%%%%%%%%%%%%%%%%%%%%%%%%
\subsection{Performance Counters}
%%%%%%%%%%%%%%%%%%%%%%%%%%%%%%%%%%%%%%%%%%%%%%%%%%%%%%%%%%%%%%%%%%
To profile the code I tried using Scorep and the Cray perftools, with varying success. The Scorep tool was useful, however it had high sampling overheads overheads. The Cray perftools caused a segmentation faults and other runtime errors errors.

To obtain reliable results without any sampling overhead, the papi-wrap\footnote{\file{github.com/bcumming/papi-wrap}} library was used. This is a simple C++ wrapper around the Papi library for hardware counters, with a simple C API for adding samplers to source code, as shown in \fig{fig:papiinline}.

%*******************************************************************************
\filelisting [C++] {./code/papi.cpp}{papi instrumentation of }{lst:papiinline}
%*******************************************************************************

\begin{myfigure}{example papi-wrap output}{fig:papisample}
\begin{center}
\begin{lstlisting} [identifierstyle=\color{Black}\small\ttfamily]
------------------------------------------------------------
collector                time     %  PAPI_DP_OPS PAPI_VEC_DP
------------------------------------------------------------
rhs_current          346.3654 45.10 152884436985 3990534760
nonvint              318.4637 41.47 118203493855 5812061816
lhs_jacob             84.7244 11.03   9945680740          0
other                  8.7107  1.13      4664371       1548
matrix_solve           4.0380  0.53   2097774870     770040
rhs_update             2.4865  0.32   1320459042          0
lhs_update             1.4597  0.19    607782930          0
update                 0.9977  0.13    749043581  239121640
lhs_cap_jacob          0.7062  0.09    546709757          0
second_order_current   0.0153  0.00         8188          0
TOTAL                767.9675
------------------------------------------------------------
\end{lstlisting}
\end{center}
\end{myfigure}

%%%%%%%%%%%%%%%%%%%%%%%%%%%%%%%%%%%%%%%%%%%%%%%%%%%%%%%%%%%%%%%%%%
\subsection{Breakdown of time to solution}
%%%%%%%%%%%%%%%%%%%%%%%%%%%%%%%%%%%%%%%%%%%%%%%%%%%%%%%%%%%%%%%%%%

%%%%%%%%%%%%%%%%%%%%%%%%%%%%%%%%%%%%%%%%%%%%%%%%%%%%%%%%%%%%%%%%%%
\subsection{Profiling With Scorep}
\label{sec:scorep}
%%%%%%%%%%%%%%%%%%%%%%%%%%%%%%%%%%%%%%%%%%%%%%%%%%%%%%%%%%%%%%%%%%
\begin{todo}
Describe steps required to get test with Scorep/Cube/Vampir.
\end{todo}

