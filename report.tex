\documentclass[11pt,a4paper]{article}

\usepackage[colorlinks=true,linkcolor=blue!30!black]{hyperref}
\usepackage[pdftex]{graphicx}
\usepackage[cmex10]{amsmath}
\usepackage[usenames,dvipsnames]{color}
\usepackage{xspace}
\usepackage{fullpage}
\usepackage{tikz}
\usepackage{pgfplots}
\usepackage{pgfplotstable}
\usepackage{textcomp}
\usepackage{multirow}
\usetikzlibrary{positioning}
\usetikzlibrary{shapes,arrows,backgrounds,fit,shapes.geometric,calc}
\usetikzlibrary{pgfplots.groupplots}
\usepackage{url}
\usepackage{listings}
\usepackage{lstautogobble}
\usepackage{framed}
\usepackage{tcolorbox}


\newcommand{\todo}[1]{\textbf{\textcolor{Blue}{TODO: #1}}}
\newcommand{\hilight}[1]{\textit{\textcolor{Red}{#1}}}
\newcommand{\lst}[1]{\lstinline!#1!}
\newcommand{\fig}[1]{{Fig.}~\ref{#1}}
\newcommand{\sect}[1]{{Section}~\ref{#1}}
\newcommand{\eq}[1]{(\ref{#1})}
\newcommand{\ap}[1]{Appendix~\ref{#1}}
\newcommand{\tbl}[1]{Tbl.~\ref{#1}}
\newcommand{\file}[1]{\lstinline[basicstyle=\normalsize,]!#1!}
\newcommand{\pder}[2]{\frac{\partial{#1}}{\partial{#2}}}
\newcommand{\neuron}{CoreNeuron\xspace}
\newcommand{\dx}{{\Delta x}}
\newcommand{\dt}{{\Delta t}}
\newcommand{\at}[1]{{#1}_{i}}
\newcommand{\atplus}[1]{{#1}_{i+1}}
\newcommand{\atminus}[1]{{#1}_{i-1}}
\newcommand{\atplushalf}[1]{{#1}_{i+\frac{1}{2}}}
\newcommand{\atminushalf}[1]{{#1}_{i-\frac{1}{2}}}
\newcommand{\vv}[1]{\mathbf{#1}}
\newcommand{\att}[1]{{#1}^{n}}
\newcommand{\attplus}[1]{{#1}^{n+1}}
\newcommand{\attplushalf}[1]{{#1}^{n+\frac{1}{2}}}
\newcommand{\hoc}{\lst{.mod}\xspace}

\tcbuselibrary{listings}
\tcbuselibrary{breakable}

% this defines a note environment
\newenvironment{note}
{\begin{tcolorbox}[colframe=blue!70!black,breakable,colback=blue!5,title=Note,fonttitle=\bfseries]}
{\end{tcolorbox}}

% takes 3 options : (file, )
\newtcbinputlisting[auto counter]{\filelisting}[3][C++]{
    listing file=#2,
    listing options={language=#1,autogobble},
    title=Listing \thetcbcounter: #3,
    fonttitle=\bfseries,
    breakable,
    colback=green!5!white,
    colframe=green!35!black,
    listing only
}

\newtcblisting[auto counter]{inlinelisting}[2]
{
    listing options={language=#1,autogobble},
    title=Listing \thetcbcounter: {#2},
    fonttitle=\bfseries,
    breakable,
    colback=green!5!white,
    colframe=green!35!black,
    listing only
}

% redefine emph for more... emphasis
\DeclareTextFontCommand{\emph}{\bfseries}

\lstdefinelanguage{julia}
{
    morekeywords = {
        if, elseif, else,
        for, while, end, in, using, local,
        type, function, return, yieldto, try, catch, error, throw, begin, quote,
        Int, Int32, Int64, Float, Float32, Float64, Array, Any
    },
    sensitive = true,
    morecomment=[l]\#,
    morestring=[b]",
    morestring=[b]',
    moredelim=*[directive]@,
    moredirectives={assert, time, eval} % macros
}[keywords,comments,strings,directives]


\lstset{
    language=[ANSI]C++,
    basicstyle=\small\ttfamily,
    identifierstyle=\color{Blue}\small\ttfamily,
    keywordstyle=\color{Red}\small\ttfamily,
    commentstyle=\color{OliveGreen}\small\ttfamily,
    breaklines=true
}
\lstset{
    emph={pw_stop_collector,
          pw_start_collector,
          pw_new_collector,
          pw_print_table
    },
    emphstyle={\color{Bittersweet}\bfseries\normalsize}
}

\definecolor{shadecolor}{rgb}{0.8,0.8,0.8}

\begin{document}

% title and cover page
\title{\neuron Overview}
\author{Ben Cumming\\CSCS -- Swiss National Supercomputing Center}
\date{\today}
\maketitle
\tableofcontents

% abstract
\abstract{
This document presents an overview of the \neuron code that was released as part of the PCP process for the Human Brain Project in July 2014. The analysis focuses on the structure of the code that has the highest computational overheads.

More detailed analysis of the algorithms themselves, and important parts of the code like spike communication, that do not contribute to the time to solution are not considered for analysis. For now I will focus on just the computationally intensive parts of the code.

Note that this analysis is based on an early version of the \neuron code, and that different algorithms (e.g. modelling plasticity) may significantly increase the importance of parts of the application not yet considered here.
}

%%%%%%%%%%%%%%%%%%%%%%%%%%%%%%%%%%%%%%%%%%%%%%%%%%%%%%%%%%%%%%
\section{What is \neuron?}
%%%%%%%%%%%%%%%%%%%%%%%%%%%%%%%%%%%%%%%%%%%%%%%%%%%%%%%%%%%%%%
The version of \neuron that was released for the PCP is derived directly from \emph{HBPNeuron}, a flavour of \emph{Neuron} maintained by the Blue Brain Project (BBP) group at EPFL. \neuron is derived directly in the sense that it is a subset of the features and corresponding code from HBPNeuron. The code has been modified as much as needed to remove it from the larger HBPNeuron infrastructure, and reduce the memory footprint of the code.

Note that HBPNeuron refers to the Neuron plus the \hoc files used to define the models used in the BBP group. The computational back end is identical for both HBPNeuron and Neuron\footnote{HBPNeuron has some additional I/O routines \lst{bbsave_state} and \lst{bbcore_write} that will be integrated back into Neuron at some point.}.

%%%%%%%%%%%%%%%%%%%%%%%%%%%%%%%%%%%%%%%%%%%%%%%%%%%%%%%%%%%%%%
\section{The Code}
The code in its current form is a mixture of C and C++.

It must be noted that the code base is currently very challenging to understand and benchmark. It is derived directly from the Bluron/Neuron code base, which has grown organically over more than 20 years. There are very many opportunities to simplify and improve the code using modern programming languages and development techniques.

To use both CPUs and accelerators (e.g. GPU and MIC) effectively, parts of the code will have to be refactored or rewritten. The main challenge in refactoring the code is not the complexity of the algorithms, it is the difficulty of understanding the current code. Developing documentation of the algorithms as they are currently implemented is an essential first step.

From early work with the code base, the majority of wall time for the example circuits release for the PCP is spent in a relatively small set of code: the \lst{nrnoc} solver implementation, and the mechanisms defined in \lst{/mech/cfiles}. Improving performance, portability and maintainability will benefit from working on the core computational API.

An aim of this report is as a first attempt at describing the algorithms clearly, to make it possible to reason about how them without getting distracted by implementation details. To assist in this, many of the algorithms are presented in a pseudo-language similar to \emph{Julia}\footnote{See the website: \file{julialang.org}}, which should be familiar to users familiar with \emph{Matlab}.

%-------------------------------------------------------------
\subsection{Understanding The Code}
%-------------------------------------------------------------
Neuron has grown organically, with new features ``bolted on'' over a long period of time (e.g. support for Python scripting, threading, etc.).
Some of these features have been removed in \neuron, leaving behind some design patterns that at first confuse newcomers.
One consequence is that there are some data structures that are overly-complicated for the role they currently serve, but they were part of a more complicated design before the \neuron rewrite.
There are also many instances where the naming of variables, global variables, and gratuitous use of the preprocessor don't help.
\begin{note}
It is recommended that the code be reformatted using a tool like \file{clang-format} (especially the automatically generated C files in \file{mech/cfiles}). Applying the preprocessor to individual files shows the origin of symbols that have been \#defined with the preprocessor. For example, using the ``\file{-E -P}'' flag in \texttt{gcc} (\lst{-E} applies the preprocessor, and \lst{-P} makes the output more human readable.)
\end{note}

%-------------------------------------------------------------
\subsection{Code Layout}
%-------------------------------------------------------------
The source code is packaged in a file \lst{CoreBluron.tar.gz}, which has the directory structure in shown below.
In terms of time to solution, functions defined in \lst{mech/cfiles} dominate. These are called from the solver routines in \lst{nrnoc}, which implements the core computation, and is the focus of the analysis here.

\begin{infobox}{The directory structure of the \neuron distributed as part of the PCP}
\begin{itemize}[leftmargin=*]
    %%%%%%%%%%%%%%%%%%%%%%%%%%%%%%%%%%%%%%%%%
    \item \textbf{nrniv}
    \begin{itemize}
        \item C and C++ (11,470 lines).
        \item The \neuron driver: \lst{main} function is in \lst{main1.cpp}.
    \end{itemize}

    %%%%%%%%%%%%%%%%%%%%%%%%%%%%%%%%%%%%%%%%%
    \item \textbf{nrnoc}
    \begin{itemize}
        \item C (2,889 lines).
        \item The \neuron ``engine''
        \begin{itemize}
            \item storage
            \item solvers
            \item time stepping
        \end{itemize}
    \end{itemize}

    %%%%%%%%%%%%%%%%%%%%%%%%%%%%%%%%%%%%%%%%%
    \item \textbf{mech/cfiles}
    \begin{itemize}
        \item C (11,301 lines).
        \item Definitions of all the mechanisms.
        \item generated from \hoc files by Neuron.
        \item the generated code is very messy (use \lst{clang-format} to make things bearable)
    \end{itemize}

    %%%%%%%%%%%%%%%%%%%%%%%%%%%%%%%%%%%%%%%%%
    \item \textbf{nrnmpi}
    \begin{itemize}
        \item C (1,096 lines).
        \item Wrappers around MPI routines.
        \item Spike exchange implementation.
        \item Global variables that store MPI state.
    \end{itemize}

    %%%%%%%%%%%%%%%%%%%%%%%%%%%%%%%%%%%%%%%%%
    \item \textbf{utils}
    \begin{itemize}
        \item C++ (4,494 lines)
        \item Random number generators
    \end{itemize}
\end{itemize}
\end{infobox}

%%%%%%%%%%%%%%%%%%%%%%%%%%%%%%%%%%%%%%%%%%%%%%%%%%%%%%%%%%%%%%
\subsection{Building}
%%%%%%%%%%%%%%%%%%%%%%%%%%%%%%%%%%%%%%%%%%%%%%%%%%%%%%%%%%%%%%
The code was built on Cray XC-30 system Piz Daint at CSCS with minimal fuss using the GNU toolchain.
The Cray compiler toolchain had problems that are not insurmountable, but they would require a lot of tinkering with the \emph{Buildyard}\footnote{\file{github.com/Eyescale/Buildyard}} build tool used by \neuron.

The Buildyard uses cmake, with a custom set of cmake modules developed by BBP (the BuildYard modules). Many of these modules are not required by \neuron, and configuration can be sped up significantly by removing them (for example the C++11 tests). The cmake configuration attempts to determine the version of the Cray compiler by passing a flag that the compiler doesn't recognise, causing the configuration to exit.

%%%%%%%%%%%%%%%%%%%%%%%%%%%%%%%%%%%%%%%%%%%%%%%%%%%%%%%%%%%%%%
\subsection{Datasets}
%%%%%%%%%%%%%%%%%%%%%%%%%%%%%%%%%%%%%%%%%%%%%%%%%%%%%%%%%%%%%%
There are two data sets provided with the PCP benchmark code:
\begin{infobox}{The two data sets available with the PCP}
\begin{itemize}[leftmargin=*]
    \item \textbf{TEST1\_CACHE} A network small enough to fit into Cache of one rack of BG/Q. Has size of 2.5G on disk.
    \item \textbf{TEST2\_DRAM} A much larger network (size 4.5T on disk), that fits in DRAM of one rack of BG/Q.
\end{itemize}
\end{infobox}

%%%%%%%%%%%%%%%%%%%%%%%%%%%%%%%%%%%%%%%%%%%%%%%%%%%%%%%%%%%%%%



\clearpage
\section{The Algorithm}
It will be shown later that the implementation of the spatial and temporal integration of the \emph{cable equation} that describes the time evolution of voltage in a cell accounts for over 99\% of all wall time. The discussion of \emph{the algorithm} here offers a basic mathematical description of the algorithm with many details that do not affect the implementation ommitted.

%%%%%%%%%%%%%%%%%%%%%%%%%%%%%%%%%%%%%%%%%%%%%%%%%%%%%%%%%%%%%%%%%%
\subsection{Discretization of The 1D Cable Equation}
%%%%%%%%%%%%%%%%%%%%%%%%%%%%%%%%%%%%%%%%%%%%%%%%%%%%%%%%%%%%%%%%%%
The partial differential equation (PDE) that describes the time evolution of the voltage at each node a discretized cell has the following general form:
\begin{equation}
     C\pder{V}{t} + I = f \pder{}{x} \left( g\pder{V}{x} \right)
\end{equation}
where $f$ and $g$ are functions of the spatial dimension $x$ (they are functions of \emph{cable radius} in this model). The value of current $I$ and capacitance $C$ are both dependent on the voltage $V$.

See \ap{appendix:discretization} for a detailed derivation of the spatial and temporal discretization. The discretization gives rise to a tridiagonal system of linear equations that is solved at every time step. The equation at each point has the following form,
\begin{equation}
    \at{a} \atplus{V}^{n+1} + \at{d} \at{V}^{n+1} + \at{b} \atminus{V}^{n+1} = \at{r}
\end{equation}
where the diagonals in the matrix are defined
\begin{align}
    \text{ upper diagonal : }  \at{a}  &=  -\frac{\at{f}\atplushalf{g}}{2\dx^2}, \nonumber \\
    \text{ lower diagonal : }  \at{b}  &=  -\frac{\at{f}\atminushalf{g}}{2\dx^2}, \nonumber \\
    \text{       diagonal : }  \at{d}  &=  \frac{\at{C}}{\dt} - ( \at{a} + \at{b} ), \nonumber \\
    \text{right hand side : }  \at{r}  &=  \frac{\at{C}}{\dt} \at{V}^n
                - \at{I}
                - \at{a} \left( \atminus{V}^{n} - \at{V}^{n} \right)
                - \at{b} \left( \atplus{V}^{n}  - \at{V}^{n} \right). \nonumber
\end{align}

The off-diagonal coefficients in the linear system, i.e. $\at{a}$ and $\at{b}$,  are constant in time because they depend only on the radius of the cable. In practice the off-diagonal entries in $\vv{a}$ and $\vv{b}$ are computed once at start up. The values on the diagonal and right hand side vector, i.e. those in $\vv{d}$ and $\vv{r}$ are updated at each time step, then modified when solving the linear system using Thomas' algorithm. The values on the diagonal and right hand side vector are calculated using the precomputed values for $\at{a}$ and $\at{b}$.
%%%%%%%%%%%%%%%%%%%%%%%%%%%%%%%%%%%%%%%%%%%%%%%%%%%%%%%%%%%%%%
\subsection{Trees and Branching}
%%%%%%%%%%%%%%%%%%%%%%%%%%%%%%%%%%%%%%%%%%%%%%%%%%%%%%%%%%%%%%
Here the one-dimensional discretization in the previous section will be extend to include branching, whereby the spatial domain is composed of a series of one-dimensional \emph{sections}, that are joined at branch points to form a tree.

A small tree structure that will be used throughout this section is illustrated in \fig{fig:tree}(a). It is important to note that the graph formed by the branching sections is a true tree, i.e. it has no circuits (once a section has branched, the branches can not ``rejoin'').

\begin{infobox}{Terms used describing discretization of trees}
\begin{itemize}[leftmargin=*]
    \item \textbf{section} a branch in the tree structure, which corresponds to the one-dimensional line segment between branch points. These are numbered $s*$ in \fig{fig:tree}(a).
    \item \textbf{branch} same definition as section.
    \item \textbf{node} a point in the spatial discretization. These are numbered in \fig{fig:tree}(b).
    \item \textbf{branch node} a node where two branches join. These are the blue points denoted $n*$ \fig{fig:tree}(a).
\end{itemize}
\end{infobox}

\begin{myfigure}{example of numbering of nodes and branches}{fig:tree}
\begin{center}
\includegraphics[width=0.6\textwidth]{./images/tree.pdf}
\\{\normalsize (a)}\\
\includegraphics[width=\textwidth]{./images/discrete_a.pdf}
\includegraphics[width=\textwidth]{./images/discrete_b.pdf}
\includegraphics[width=\textwidth]{./images/discrete_c.pdf}
\includegraphics[width=\textwidth]{./images/discrete_d.pdf}
\includegraphics[width=\textwidth]{./images/discrete.pdf}
\\{\normalsize (b)}
\end{center}
(a) The high level branch and connection numbering scheme, with the branch nodes and sections numbered; (b) the numbering of individual nodes in the fully discretized domain with 5 segments per section.
\end{myfigure}

The first step of the spatial discretization is to discretize each one-dimensional section. Then the nodes are numbered using a scheme that gives the matrix a sparsity structure that allows the linear system to be solved in linear time, equivalent to Thomas algorithm.

A numbering scheme for the nodes that facilitates efficient solution of the linear system is not unique. A general description of a valid numbering strategy is as follows:
\begin{enumerate}
    \item
        One node is chosen as the \emph{root node} of the tree, either a terminal node or a branch node, and assigned index 1.
    \item
        The tree is then traversed along each branch with the root node as the starting point. The nodes are numbered sequentially in ascending order.
    \item
        Every node (except the root node) has one-and-only-one \emph{parent node}, which is its neighbour that is closer to the root node (and thus has a lower index).
\end{enumerate}
The key property that is maintained by the numbering scheme is that the index of a node's parent node is lower than the node's index (equivalent to \emph{minimum degree ordering}). An example of a numbering strategy being recursively applied to our example tree is shown in \fig{fig:tree}(b).

\begin{myfigure}{matrix sparsity pattern}{fig:sparsity}
\begin{center}
\includegraphics[width=0.5\textwidth]{./images/sparsity.pdf}
\end{center}
Sparsity pattern of the matrix corresponding to the tree numbering in \fig{fig:tree}.
\end{myfigure}

To describe the sparsity pattern of the matrix from this numbering only the parent indexes, $p_i\quad i\in[2:n]$, for each node need to be stored. The matrix sparsity pattern, which is illustrated for our example tree numbering in \fig{fig:sparsity}, has the following properties:
\begin{infobox}{Properties of the linear system}
\begin{itemize}[leftmargin=*]
    \item
        The sparsity pattern is symmetric.
    \item
        The diagonal values are all nonzero.
    \item
        There is exactly one off diagonal value in each row of the lower triangle at $a_{ik}$ and exactly one off diagonal value in each column of the upper triangle at $a_{ki}$, where $k=p_i$.
    \item
        The matrix can be stored with three vectors, $\vv{a}$, $\vv{b}$ and $\vv{c}$:
        \begin{align}
            a_i &= A_{ki} \\
            b_i &= A_{ik} \\
            d_i &= A_{ii}
        \end{align}
        where $k=p_i$.
\end{itemize}
\end{infobox}

\begin{note}
The choice of the root node and the traversal order when generating the numbering is important if trying to solve the linear system in parallel. It is possible to solve sub-trees that branch from a branch node independently, before combining the results to solve the value at the branch node. For sequential solution on the CPU this isn't important, however a GPU implementation might try to take advantage of this.
\end{note}
%%%%%%%%%%%%%%%%%%%%%%%%%%%%%%%%%%%%%%%%%%%%%%%%%%%%%%%%
\subsection{Hines Algorithm}
\label{sec:hines}
%%%%%%%%%%%%%%%%%%%%%%%%%%%%%%%%%%%%%%%%%%%%%%%%%%%%%%%%
These linear systems can be solved very efficiently, in linear $O(N)$ time, using an algorithm that is equivalent to the Thomas algorithm for solving tri-diagonal systems. This algorithm, called Hines algorithm, proceeds by eliminating the nonzero entries in the upper triangle of $A$. Recall the matrix property that there is only one non-zero value in each column of the upper triangle at $A_{ki}$, which is stored in $a_i$.
\begin{equation*}
\left(
        \begin{array}{ccc}
            A_{kk} & \dots      & A_{ki} \\
        \vdots     & \ddots     & \mathbf{0} \\
            A_{ik} & \mathbf{0} & A_{ii}
        \end{array}
\right)
\text{which is stored as:}
\left(
        \begin{array}{ccc}
            d_k & \dots      & a_i \\
        \vdots  & \ddots     & \mathbf{0} \\
            b_i & \mathbf{0} & d_i
        \end{array}
\right)
\end{equation*}
The nonzero value in column $i$, i.e. $a_i$, can be zeroed out with a row operation
\begin{equation*}
    \text{row}_k \leftarrow \text{row}_k - a_i/d_i\cdot\text{row}_i.
\end{equation*}
In practice the value of $a_i$ is not changed, and only the values in on the diagonal and in the RHS vector have to be modified
\begin{align}
d_k \leftarrow d_k - a_i/d_i\cdot b_i, \nonumber \\
r_k \leftarrow r_k - a_i/d_i\cdot r_i, \nonumber
\end{align}
So that our submatrix now looks like
\begin{equation*}
\left(
        \begin{array}{ccc}
            d_k - a_i/d_i\cdot b_i  & \dots      & \mathbf{0} \\
        \vdots                      & \ddots     & \mathbf{0} \\
            b_i                     & \mathbf{0} & d_i
        \end{array}
\right)
\left(
        \begin{array}{c}
            r_k - a_i/d_i\cdot r_i \\
        \vdots                     \\
            r_i
        \end{array}
\right)
\end{equation*}


The algorithm proceeds by eliminating the values in the upper triangle with a backward sweep over columns $n:-1:2$.
A forward sweep is then used to eliminate the nonzeros in the lower triangle, determine the solution.

\begin{note}
In practice, this algorithm is very efficient, contributing less than 1\% to the time to solution in the benchmarks released with the PCP. 
\end{note}




\clearpage
\section{The Time Step}
In this section an will look at how the main time-stepping algorithm, which accounts for nearly all the time to solution, is implemented. However, before looking at the time step implementation, a short description of all of the main stages of the simulation is presented to put the time stepping code in context.
%%%%%%%%%%%%%%%%%%%%%%%%%%%%%%%%%%%%%%%%%%%%%%%%%%%%%%%%%%%%%%
\subsection{Overview of Main}
%%%%%%%%%%%%%%%%%%%%%%%%%%%%%%%%%%%%%%%%%%%%%%%%%%%%%%%%%%%%%%
The driver code is in \file{nrniv/main1.cpp}. From the breakdown of the wall time for the TEST2 data set in \tbl{tbl:wallmain} it is apparent that from a computational point of view, the only component of importance is the time stepping/sover portion of the program in \lst{BBS_netpar_solve}, which takes 99.8\% of time to solution.

Nevertheless, I will breifly describe the other steps performed in the main driver:
\begin{enumerate}
\item \lst{mk_mech}\\
The mechanisms are configured. A text file with a tuple for each mechanism (name, unique index, parameter count, type,  etc \dots) is scanned. This text file is generated when the cell group files are generated by HBPNeuron. This information provides a bridge between the runtime and the mechanisms implemented using the Neuron \hoc language.
\item \lst{mk_netcvode}\\
    Creates a new \lst{NetCvode} object (see \file{nrnoc/netcvod.h/cpp}). This sets up the priority queue use to send and deliver spiking events.
\item \lst{nrn_setup}\\
    Before calling \lst{nrn_setup}, the configuration file \file{files.dat} is read to see how many and which cells are to be loaded for simulation. The cells are assigned in a round-robin fashion between the MPI ranks. Then \lst{nrn_setup} is called with a list of cell ids, to load the cell data from disk.
\item \lst{BBS_netpar_mindelay} and \lst{mk_spikevec_buffer}\\
    The mindelay and spike buffer size are configured. All neuron cells can be integrated-in-time independently for the \emph{minimum network connection delay}, i.e. spikes do not have to be delivered in the interval in which they were generated. These interval boundaries are used as synchronization points.
\item \lst{BBS_netpar_solve} \\
    The time stepping code. The focus of this report.
\item \lst{output_spikes} \\
    Write spike information to disk.
\end{enumerate}

%-------------------------------------------------------------------------------
\begin{table}[htp!]
    \centering
%-------------------------------------------------------------------------------
\begin{tabular}{lrr}
\hline
section                    &    wall time (s) & contribution \% \\
\hline
\lst{mk_mech}            &    0.01   &    0.0\\
\lst{mk_netcvode}        &    0.00   &    0.0\\
\lst{nrn_setup}          &    0.69   &    0.2\\
mindelay/spike buffer      &    0.15   &    0.0\\
\lst{BBS_netpar_solve}   &    388.93 &   99.8\\
\lst{output_spikes}      &    0.01   &    0.0\\
\hline
\end{tabular}
%-------------------------------------------------------------------------------
\label{tbl:wallmain}
\caption{Breakdown of wall time for TEST2 data set running on one node of Piz Daint, with 1 cell per core.}
\end{table}
%-------------------------------------------------------------------------------

%%%%%%%%%%%%%%%%%%%%%%%%%%%%%%%%%%%%%%%%%%%%%%%%%%%%%%%%%%%%%%
\subsection{Drilling Down to The Time Step}
%%%%%%%%%%%%%%%%%%%%%%%%%%%%%%%%%%%%%%%%%%%%%%%%%%%%%%%%%%%%%%
The time stepping and all computation associated with it are performed in the \lst{BBS_netpar_solve} routine. A backtrace of the call tree from \lst{BBS_netpar_solve} to \lst{nrn_fixed_step_thread}, where 100\% of the computation is performed, is shown in \fig{fig:bbsnetpar}. The exact role played by each of these routines is not important now: some of them , and others are wrappers for passing cell groups to a pthread to separate integration (see \sect{sec:data} for more information about cell groups and threads). It is the integration of individual cell groups inside the lowest level function, \lst{nrn_fixed_step_thread} that interests us.

\begin{figure}[htp!]
\centering
\includegraphics[width=\textwidth]{./images/bbs_netpar_solve.pdf}
\caption{backtrace to the main computational routine.}
\label{fig:bbsnetpar}
\end{figure}

Each MPI rank has a set of cells assigned to it in a round robin fashion during the initialization phase (in the call to \lst{nrn_setup} in \lst{main}).
The cells are packaged together into \emph{cell groups}, with one cell group per intput file. The selection of cells in a cell group is chosen when the input files are generated to improve load balancing (static load balancing).
For more information, see \sect{sec:data}.

%The cells are then assigned to a \lst{NrnThread} on each MPI rank. An abreviated definition of is given in \fig{lst:NrnThread}

%\begin{figure}
%\begin{shaded}
%\begin{lstlisting}
%struct NrnThread {
%  // list of mechanisms
%  NrnThreadMembList *tml;
%
%  int ncell;        // number of cells
%  int end;          // number of segments
%  double *_data;    // data for all segment values
%  ...               // other data fields
%
%  // arrays holding matrix system values
%  double *_actual_rhs;
%  double *_actual_d;
%  double *_actual_a;
%  ...
%  // area of each segment
%  double *_actual_area;
%  // parent index of segments
%  int *_v_parent_index;
%};
%\end{lstlisting}
%\end{shaded}
%\label{lst:NrnThread}
%\caption{The definition of the \lst{NrnThread} data type from \file{nrnoc/multicore.h}. Note that many data members have been removed, with just some fields of interest included.}
%\end{figure}

%*******************************************************************************
\begin{figure}[htp!]
\centering
\includegraphics[width=\textwidth]{./images/calltree.pdf}
\caption{Calltree and percentage of wall time contribution for the main computational algorithm. Branches marked in blue indicate a significant contribution to wall time.}
\label{fig:calltree}
\end{figure}
%*******************************************************************************


In this section the implementation of the code that forms and solves the matrix, which accounts for 99\% of the time to solution will be described. The algorithm themselves are quite simple, however their implementation is very difficult to understand. To make it easier to understand to understand implementation, the core routines have been rewritten in a high-level pseudo code similar to Julia.

An example of the pseudo code represents the following C code
\begin{shaded}
\begin{lstlisting} [breaklines=true]
for (tml = _nt->tml; tml; tml = tml->next)
  if (memb_func[tml->index].current) {
    mod_f_t s = memb_func[tml->index].current;
    (*s)(_nt, tml->ml, tml->index);
  }
\end{lstlisting}
\end{shaded}
\noindent as
\begin{shaded}
\begin{lstlisting} [language=julia,breaklines=true]
for mechanism in thread.mechanisms
  mechanism.current(thread, mechanism.data)
end
\end{lstlisting}
\end{shaded}
\noindent There is a trade-off, whereby idiomatic Julia code would look like the following, but it is kept in the form above to more closely match the data flow in the C code
\begin{shaded}
\begin{lstlisting} [language=julia,breaklines=true]
for mechanism in thread.mechanisms
  current(mechanism)
end
\end{lstlisting}
\end{shaded}
\noindent Not that a similar level of clarity would be possible with well-designed C+11 code.

The inner part of each time step is implemented in the function \lst{nrn_fixed_step_thread()}, in \file{nrnoc/fadvance_core.c}. The routine takes as its argument a pointer to a struct of type \lst{NtnThread}, see \fig{lst:NrnThread}, which holds state relating to a set of cells to be integrated in time.
\begin{shaded}
\lstinputlisting [language=julia,breaklines=true] {./code/fixed_step_thread.jl}
\end{shaded}

A breakdown of wall time for the steps in \lst{nrn_fixed_step()} is given in \fig{fig:calltree}. Some of the routines listed here have less than 1\% of wall time (including the linear system solve in \lst{nrn_solve_minimal()}), however they are discussed below because they access they have implementation details that will influence the implementation on many-core architectures (e.g. GPU and MIC).

%%%%%%%%%%%%%%%%%%%%%%%%%%%%%%%%%%%%%%%%%%%%%%%%%%%%%%%%%%%%%%
\subsubsection{Building matrix and RHS: \lst{setup_tree_matrix()}}
%%%%%%%%%%%%%%%%%%%%%%%%%%%%%%%%%%%%%%%%%%%%%%%%%%%%%%%%%%%%%%
The function \lst{setup_tree_matrix()} generates the diagonal, the $d_i$ values, and the RHS vector. These tasks are performed in two separate routines, \lst{nrn_lhs()} and \lst{nrn_rhs()}.
\begin{shaded}
\lstinputlisting [language=julia,breaklines=true] {./code/setup_tree_matrix.jl}
\end{shaded}

Points
\begin{itemize}
\item
    The array \lst{p} is an index array containing the parent node indexes.
\item
    The arrays \lst{VEC_*} correspond to the vectors $\vv{a}, \vv{b}, \vv{d}, \vv{v}, \vv{r}$ that define the linear system.
\item
    Each thread has multiple cells, each with their own tree representation. The cells are packed together, with the root node of each cell placed first in the list of all nodes, hence the definition of \lst{child_nodes} excluding indexes $1:ncells$.
\item
    Nearly all (i.e. 99\%) of the time in these two routines is spent in the calls to the \lst{mechanism.current()} and \lst{mechanism.jacob()} routines.
\item
    The matrix updates still must be considered, because there are potential race conditions in a multi-threaded/GPU implementation. For example the statement \lst{VEC_RHS[p[i]] += dv * VEC_A[i]} will lead to a race condition if two threads with the same parent node try to update the RHS vector at the same time.
\end{itemize}

The \lst{mechanism.current()} and \lst{mechanism.jacob()} routines are defined in the \file{/mech/cfiles} path, and are automatically generated from Neuron hoc DSL. \fig{fig:calltree} shows that all of the computational work in the \neuron benchmark used in this report is performed by functions from the hoc layer.



\clearpage
\section{Data Layout}
\label{sec:data}
%%%%%%%%%%%%%%%%%%%%%%%%%%%%%%%%%%%%%%%%%%%%%%%%%%%%%%%%%%%%%%%%%%%%%%%%%%%%%%
\subsection{Parallel Data Distribution}
%%%%%%%%%%%%%%%%%%%%%%%%%%%%%%%%%%%%%%%%%%%%%%%%%%%%%%%%%%%%%%%%%%%%%%%%%%%%%%
\begin{itemize}
\item
    Each neuron cell is represented as of a tree of nodes, as illustrated in \fig{fig:tree}.
\item
    The properties of individual cells vary significantly, so that the computational resources required to process cells varies.
\item
    The biggest cause of this difference is the distribution of nodes with different mechanism types.
\item
    The \lst{states}/\lst{rates}/\lst{current} kernels of some mechanisms have much higher arithmetic intensity than others.
\item
    The combination of mechanisms in a cell, which varies between different cell types, can be used a-priori estimate it's computational complexity.
\end{itemize}

\begin{itemize}
\item
    To ensure load balancing, the cells are grouped into groups that have roughly equivalent computational overheads during circuit generation.
\item
    Some groups have more cells than others to ensure that total computational effort required per group is balanced.
\item
    Each group of cells is stored in a separate \file{.dat} file, which are then distributed in a round-robin fashion when a \neuron simulation is started.
\end{itemize}

\begin{itemize}
\item
    Parallelism is implemented distributing the cells, with individual cells processed serially.
\item
    There are two levels of parallelism:
    \begin{enumerate}
    \item
        \textbf{MPI}: the cells are distributed between MPI ranks.
    \item
        \textbf{thread}: the cells on each MPI rank are then assigned to a \emph{thread}.
    \end{enumerate}
\item
    Each thread stores the cells assigned to it in a \lst{NrnThread} data structure (see \fig{lst:NrnThread})
\item
    There is one \lst{NrnThread} data structure for each thread.
\item
    Threading is performed using pthreads, with explicit communication of spike information between threads and between MPI processes.
\end{itemize}

%%%%%%%%%%%%%%%%%%%%%%%%%%%%%%%%%%%%%%%%%%%%%%%%%%%%%%%%%%%%%%%%%%%%%%%%%%%%%%
\subsection{Thread Storage}
%%%%%%%%%%%%%%%%%%%%%%%%%%%%%%%%%%%%%%%%%%%%%%%%%%%%%%%%%%%%%%%%%%%%%%%%%%%%%%
Overview of storage of cells in one thread:
\begin{itemize}
\item
    Each thread has multiple cells assigned to it.
\item
    For the TEST2 dataset:
    \begin{itemize}
    \item
        there are around 60--70 cells per thread.
    \item
        each cell has of the order 400--450 nodes.
    \item
        each thread has 25,000--30,000 nodes.
    \end{itemize}
\item
    The nodes for all cells are stored in one flat array
    \begin{itemize}
    \item
        Given have $n_c$ cells and a total of $n$ nodes, the root nodes are indexed \lst{[1:n_c]}, and the rest of the nodes are indexed \lst{[n_c+1:n]}.
    \item
        This is evident in the loops over \lst{child_nodes=ncells+1:nnondes} and \lst{1:ncells}
    \end{itemize}
\end{itemize}

\noindent
Mechanisms in Neuron are implemented using the \hoc DSL, which is translated into C code.
\neuron has the C files allready translated from the \hoc files that are used by BBP\footnote{This reduces the complexity of \neuron, decoupling \neuron from Neuron, which will make it easier modify how the mechanisms are defined.}.
The translated mechanism definitions are in \file{mech/cfiles}, with one mechanism per C source file.
Each file defines functions (like \lst{jacob}, \lst{current}, \lst{alloc}) and meta-data (such as the number of variables required to store a mechanism's state for a node in the tree).
\begin{itemize}
\item
    The mechanism data is stored in global two arrays: \lst{Memb_list memb_list[]} and \lst{Memb_func memb_func[]}.
    \begin{enumerate}
    \item
        The \lst{Memb_func} type has function pointers to the \lst{jacob}, \lst{current}, \lst{state} and other mechanism-specific functions, and other meta-data specific to the mechanism.
    \item
        The \lst{Memb_list} has a pointer to the per-node data, and a list of all the nodes that the mechanism is defined for.
    \end{enumerate}
\item
    The implementation of a mechanism in \lst{mech/cfiles} provides has a function \lst{??} that calls the \lst{register_mech()} function, which adds the mechanisms function callbacks for \lst{jacob} etc, along with meta-data into the global arrays (see \fig{lst:register_mech}).
\item
    Each thread has a list of mechanisms assigned to it, which are accessed via a linked list \lst{NrnThread::mechanisms} (see \fig{lst:NrnThreadInfo} where I have changed the name \lst{tml} to \lst{mechanisms}, to better match the pseudo code.) The linked list indexes the global arrays \lst{memb_list} and \lst{memb_func} \hilight{(why not use an array instead of a linked list?)}.
\item
    There are many opportunities to improve the interface between the runtime (solvers etc) and user-defined mechanisms.
    This model is well-suited to standard object-oriented design.
    Furthermore, much of the meta-data that is currently passed as runtime parameters could be stored as type-information that could help the compiler optimize more agressively.
\end{itemize}

\noindent
Mechanisms and their storage:
\begin{itemize}
\item
    All mechanisms are not applied to different nodes. For example, the \lst{ProbAMPANDMDA_EMS} mechanism will only be applied at a subset of the nodes in a cell.
\item
    Each mechanism has ``state'' that is stored for each node to which it is applied. This state is a set of double-precision values (e.g. a set of values describing the time evolution of a ordinary differential equation).
\item
    The number of state variables varies between mechanisms, ranging from 3 to 35 values.
\item
    Each entry in \lst{memb_list[]} stores 
    \begin{itemize}
    \item
        \lst{int nodecount}: the number of nodes to which
    \item
        \lst{int nodeindices[nodecount]}: the indexes of the nodes to which the mechanism is to be applied.
    \item
        \lst{double data[nodecount*var_per_node]}: AoS storage for the mechanism  values.
    \end{itemize}
\end{itemize}


\begin{figure}
\begin{shaded}
\lstinputlisting [language=C++,breaklines=true] {./code/storage.cpp}
\end{shaded}
\label{lst:NrnThreadInfo}
\caption{The thread (\lst{NrnThread}), mechanism data (\lst{Memb_list}) and mechanism functionality (\lst{Memb_func}) types. I have removed and renamed much of the members to make them better match the pseudo code. Some C++ coding style has also been used.}
\end{figure}

\begin{figure}
\begin{shaded}
\lstinputlisting [language=C++,breaklines=true] {./code/register_mech.cpp}
\end{shaded}
\label{lst:register_mech}
\caption{The routines used in registering a mechanism}
\end{figure}

\begin{figure}[htp!]
\tikzset{
    %Define standard arrow tip
    >=stealth',
    % Define arrow style
    pil/.style={
           ->,
           thick,
           shorten <=2pt,
           shorten >=2pt,}
}
\newcommand{\nodewidth}{1.1cm}
\newcommand{\nodeheight}{0.5cm}
\begin{tikzpicture}[x=0cm, y=0cm, node distance=0 cm,outer sep = 0pt]
\tikzstyle{t0}=[draw, rectangle,  minimum height=\nodeheight, minimum width=\nodewidth, fill=red!10,anchor=south west]
\tikzstyle{t1}=[draw, rectangle,  minimum height=\nodeheight, minimum width=\nodewidth, fill=green!10,anchor=south west]
\tikzstyle{t2}=[draw, rectangle,  minimum height=\nodeheight, minimum width=\nodewidth, fill=blue!10,anchor=south west]
\tikzstyle{t3}=[draw, rectangle,  minimum height=\nodeheight, minimum width=\nodewidth, fill=yellow!20,anchor=south west]
\tikzstyle{index}=[draw, rectangle,  minimum height=\nodeheight, minimum width=\nodewidth, fill=black!10,anchor=south west]
\tikzstyle{blank}=[draw=none, fill=none, rectangle,  minimum height=\nodeheight, minimum width=1cm, anchor=south west]
\tikzstyle{over}=[draw=black, rectangle,  minimum height=\nodeheight, minimum width=3.3cm, anchor=south west]
\tikzstyle{pointer}=[draw=black!50, rectangle,  minimum height=\nodeheight, minimum width=2.5cm, anchor=south west]

\node[pointer] (dataAoS) at (0,0)         {\lst{dataAoS}};
\node[t0] (m00) [right = 1cm of dataAoS] {$m_{0,0}$}
    edge[pil,<-] (dataAoS.east);
\node[t1] (m01) [right = of m00] {$m_{0,1}$};
\node[t2] (m02) [right = of m01] {$m_{0,2}$};
\node[over] (over0) [below = of m01] {node 0 params};
\node[t0] (m10) [right = of m02] {$m_{1,0}$};
\node[t1] (m11) [right = of m10] {$m_{1,1}$};
\node[t2] (m12) [right = of m11] {$m_{1,2}$};
\node[over] (over1) [below = of m11] {node 4 params};
\node[t0] (m20) [right = of m12] {$m_{2,0}$};
\node[t1] (m21) [right = of m20] {$m_{2,1}$};
\node[t2] (m22) [right = of m21] {$m_{2,2}$};
\node[blank] (blank) [right = of m22] {...};
\node[over] (over2) [below = of m21] {node 11 params};
\node[t0] (m30) [right = of blank] {$m_{n,0}$};
\node[t1] (m31) [right = of m30] {$m_{n,1}$};
\node[t2] (m32) [right = of m31] {$m_{n,2}$};\\
\node[over] (over3) [below = of m31] {node 87 params};

\node[index] (i0) [above = 0.5 cm of m00]   {$0$};
\node[index] (i1) [right = of i0]   {$4$};
\node[index] (i2) [right = of i1]   {$11$};
\node[blank] (blanki) [right = of i2] {...};
\node[index] (in) [right = of blanki]   {$87$};
\node[pointer] (indicesAoS) [left = 1cm of i0]   {\lst{nodeindices}}
    edge[pil,->] (i0.west);

\node[index] (nnodes) [above = 0.5 cm of i0]   {$n$};
\node[pointer] (numnodes) [left = 1cm of nnodes]   {\lst{nodecount}}
    edge[pil,-] (nnodes.west);

\node[t0]  (00) [below = 1 cm of m00] {$m_{0,0}$};
\node[t0]  (10) [right = of 00] {$m_{1,0}$};
\node[t0]  (20) [right = of 10] {$m_{2,0}$};
\node[blank] (blank0) [right = of 20] {...};
\node[t0]  (n0) [right = of blank0] {$m_{n,0}$};

\node[t1]  (01) [below = 1.7 cm of m00] {$m_{0,1}$};
\node[t1]  (11) [right = of 01] {$m_{1,1}$};
\node[t1]  (21) [right = of 11] {$m_{2,1}$};
\node[blank] (blank1) [right = of 21] {...};
\node[t1]  (n1) [right = of blank1] {$m_{n,1}$};

\node[t2]  (02) [below = 2.4 cm of m00] {$m_{0,2}$};
\node[t2]  (12) [right = of 02] {$m_{1,2}$};
\node[t2]  (22) [right = of 12] {$m_{2,2}$};
\node[blank] (blank2) [right = of 22] {...};
\node[t2]  (n2) [right = of blank2] {$m_{n,2}$};

\node[pointer] (indicesSoA) [left = 1cm of 01]   {\lst{dataSoA[]}}
    edge[pil,->,anchor=east] (00.west)
    edge[pil,->] (01.west)
    edge[pil,->] (02.west);

\end{tikzpicture}

\caption{The current AoS layout of mechanism parameters for all applicable nodes.}
\end{figure}

\noindent
Observations
\begin{itemize}
\item
    Splitting the cells into seperate, thread-specific, data structures complicates the code. This feels like it was added at some point to facilicate threading.
\item
    Could be stripped away, to store all cells on a node/numa-region/device into a single pool.
\item
    The AoS storage is inefficient:
    \begin{itemize}
    \item
        It doesn't vectorize (see \fig{fig:papisample}).
    \item
        Poor cache/bandwidth utlization for loops (such as the \lst{jacob} update) where only one or two data values in a mechanism are touched. For each 64 byte cache line loaded, only 8 bytes of 64 are used.
    \end{itemize}
    An SoA storage would address both issues.
\item
    With SoA vectorization of most loops would still not be possible, because of the gather/scatter implicit in using the node indexes to read/write to the V and RHS vectors.
    \begin{itemize}
    \item
        Perform scatter before computing the current, then gather after.
    \end{itemize}
\end{itemize}



\section{Benchmarking}
%-------------------------------------------------------------------------------
\begin{table}[htp!]
    \centering
%-------------------------------------------------------------------------------
\begin{tabular}{l|l|rrrrr}
\multirow{2}{*}{}
& cells-per-core  &    1  & 2     & 4     & 8  & 16  \\
& cells-per-node  &    8  & 16    & 32    & 64 & 128 \\
\hline
\multirow{8}{*}{nodes}
&1                & 390.8 & 383.6 & 381.6 & 380.5 & 381.3\\
&2                & 390.9 & 385.1 & 384.0 & 385.3 & 382.5\\
&4                & 394.8 & 392.7 & 389.9 & \textcolor{blue}{194.1} & 382.6\\
&8                & 400.4 & 392.8 & 385.9 & 382.7 & 381.0\\
&16               & 401.9 & 394.3 & 388.6 & 384.2 & 381.8\\
&32               & 401.7 & 393.6 & 387.0 & 383.8 & \textcolor{red}{DNF}\\
&64               & 403.0 & 395.7 & 390.0 & 385.4 & 382.3 \\
&128              & 411.0 & 396.1 & 390.4 & 385.5 & 378.7 \\
&256              & 411.4 & 397.0 & 390.7 & \textcolor{blue}{195.5} & 374.9\\
&512              & 413.5 & 396.8 & 389.2 & 377.7 & --\\
%\hline
\end{tabular}

%-------------------------------------------------------------------------------
\label{tbl:test2scaling}
\caption{Wall time for TEST2 as both the number of nodes and cells-per-node are varied. Each test always use 8 cores/MPI ranks per-node. The times in \textcolor{blue}{blue} indicate unexplained timings, and those marked \textcolor{red}{DNF} did not finish due to memory restrictions (it is not possible to run with more than 16 cells per node due to memory restrictions.)}
\end{table}
%-------------------------------------------------------------------------------

\todo{describe adding Papi counters to instrument parts of code. Show basic timing results in \fig{fig:calltree}.}

\begin{figure}[htp!]
\centering
\includegraphics[width=\textwidth]{./images/calltree.pdf}
\caption{Calltree and percentage of wall time contribution for the main computational algorithm. Branches marked in blue indicate a significant contribution to wall time.}
\label{fig:calltree}
\end{figure}



%%%%%%%%%%%%%%%%%%%%%%%%%%%%%%%%%%%%%%%%%%%%%%%%%%%%%%%%%%%%%%%%%%%%%%%%%%%%%%%%%%%
%   appendices
%%%%%%%%%%%%%%%%%%%%%%%%%%%%%%%%%%%%%%%%%%%%%%%%%%%%%%%%%%%%%%%%%%%%%%%%%%%%%%%%%%%
\clearpage
\appendix
%%%%%%%%%%%%%%%%%%%%%%%%%%%%%%%%%%%%%%%%%%%%%%%%%%%%%%%%%%%%%%%%%%%%%%%%%%%%%%%%%%%
%\newpage
\section{Discretization Details}
\label{appendix:discretization}
%%%%%%%%%%%%%%%%%%%%%%%%%%%%%%%%%%%%%%%%%%%%%%%%%%%%
\subsection{One Dimensional Discretization}
%%%%%%%%%%%%%%%%%%%%%%%%%%%%%%%%%%%%%%%%%%%%%%%%%%%%
\hilight{Francesco, this would be a good spot to put any extra details.}

The spatial and temporal discretization used in Neuron is finite difference with a Crank Nicholson time stepping scheme. The time stepping scheme has an odd form, that differs slightly from the scheme presented here. However, the exposition here adequately characterizes the code.

\emph{Finite Difference Methods in Heat Transfer, Second Edition} By Necati Ozisik (1994) gives the following finite difference discretization
\begin{align}
    \dx \left(\at{C} \frac{\attplus{\at{V}} - \att{\at{V}}}{\dt} + \attplushalf{\at{I}} \right)
    &=\nonumber \\
    & \at{f} \theta
            \left[
                \atminushalf{g} \frac{\atminus{V}^{n+1}-\at{V}^{n+1}}{\dx}
              + \atplushalf{g}  \frac{\atplus{V}^{n+1}-\at{V}^{n+1}}{\dx}
            \right] \nonumber \\
            + & \at{f} (1-\theta)
            \left[
                \atminushalf{g} \frac{\atminus{V}^{n}-\at{V}^{n}}{\dx}
              + \atplushalf{g}  \frac{\atplus{V}^{n}-\at{V}^{n}}{\dx}
            \right]
        \label{eq:cableDiscretization}
\end{align}

where the parameter $\theta$ can be used to determine the time stepping scheme ($\theta=0$ explicit, $\theta=1$ implicit, $\theta=1/2$ Crank-Nicholson).

Note that the quantity $I$ depends on the value of $V$. These values have to be evaluated to form coefficients in the linear system that is to solved at each time step. The approach taken by Neuron is the evaluate them at a half time step value at $t+\dt/2$, denoted $\attplushalf{\at{I}}$, by implicitly solving for a half step, then performing an explicit half step.

With a Crank Nicholson time stepping scheme, i.e. $\theta=1/2$, the terms on the right hand side of \eq{eq:cableDiscretization} simplify to
\begin{align}
        \frac{\at{f}}{2}
        &\left\{
            \left[
                \atminushalf{g} \frac{\atminus{V}^{n+1}-\at{V}^{n+1}}{\dx}
              + \atplushalf{g}  \frac{\atplus{V}^{n+1}-\at{V}^{n+1}}{\dx}
            \right]
        \right. \nonumber \\
        +
        &\left.
            \left[
                \atminushalf{g} \frac{\atminus{V}^{n}-\at{V}^{n}}{\dx}
              + \atplushalf{g}  \frac{\atplus{V}^{n}-\at{V}^{n}}{\dx}
            \right]
        \right\} \label{eq:crankNichRHS}
\end{align}

Substituting the spatial discretization in~\eq{eq:crankNichRHS} into~\eq{eq:cableDiscretization} and rearranging gives a tri-diagonal linear system with the form
\begin{equation}
    \at{a} \atplus{V}^{n+1} + \at{d} \at{V}^{n+1} + \at{b} \atminus{V}^{n+1} = \at{r}
\end{equation}
where the diagonals in the matrix are defined
\begin{align}
    \at{a}  &=  -\frac{\at{f}\atplushalf{g}}{2\dx^2} \\
    \at{b}  &=  -\frac{\at{f}\atminushalf{g}}{2\dx^2} \\
    \at{d}  &=  \frac{\at{C}}{\dt} + \frac{f}{2\dx^2}\left[\atminushalf{g}+\atplushalf{g}\right] \nonumber\\
            &=  \frac{\at{C}}{\dt} - ( \at{a} + \at{b} ) \\
\intertext{and}
    \at{r}  &=  \frac{\at{C}}{\dt} \at{V}^n - \at{I} + \frac{\at{f}}{2\dx^2}
                    \left[
                        \atminushalf{g} \left( \atminus{V}^{n} - \at{V}^{n} \right)
                        +
                        \atplushalf{g}  \left( \atplus{V}^{n}  - \at{V}^{n} \right)
                    \right] \nonumber\\
            &=  \frac{\at{C}}{\dt} \at{V}^n
                - \at{I}
                - \at{a} \left( \atminus{V}^{n} - \at{V}^{n} \right)
                - \at{b} \left( \atplus{V}^{n}  - \at{V}^{n} \right)
\end{align}

The off-diagnoal coefficients in the linear system, i.e. $\vv{a}$ and $\vv{b}$, are constant in time, and can be computed once at start up. The values on the diagonal and right hand side vector, i.e. $\vv{d}$ and $\vv{r}$, are computed every time step, and overwritten when solving the linear system using Thomas' algorithm. The values stored in $\vv{a}$ and $\vv{b}$ are used when recomputing them.

\hilight{The current, $I$, also makes a contribution to the diagonal due to it's dependence on $V$, which I have not included here.}


%%%%%%%%%%%%%%%%%%%%%%%%%%%%%%%%%%%%%%%%%%%%%%%%%%%%%%%%%%%%%%%%%%%%%%%%%%%%%%%%%%%
%\newpage
%\section{Source code for Hines algorithm}
%\label{appendix:hinessource}
%\todo{describe code}
\begin{shaded}
\lstinputlisting [language=julia,breaklines=true] {./code/hines.jl}
\end{shaded}


%%%%%%%%%%%%%%%%%%%%%%%%%%%%%%%%%%%%%%%%%%%%%%%%%%%%%%%%%%%%%%%%%%%%%%%%%%%%%%%%%%%

\end{document}
